\maketitle
\byline{Berkenalan dengan theCrag}{Y-Climber}
\begin{figure}[H]
    \centering
    \includegraphics[width=0.5\linewidth]{image/asset/Logo-theCrag/theCrag-white.png}
\end{figure}

\begin{multicols}{3}
\section{About theCrag}
Situs web \href{https://www.thecrag.com/}{thecrag.com} pertama kali diluncurkan pada tahun 1999 di Australia oleh Simon Dale dan Campbell Gome. Awalnya, situs ini berfungsi sebagai sumber informasi pendakian di Arapiles dan kemudian meluas ke seluruh Australia. Meskipun sempat dua kali berada di ambang kebangkrutan, tim pendiri bertekad untuk terus maju dengan proyek ini.

Titik balik penting terjadi pada tahun 2010 ketika Brendan Heywood bergabung dengan theCrag, membawa serta ide-ide segar, keahlian, dan semangat baru. Sepanjang 10 tahun berikutnya, keahlian teknisnya, serta upayanya dalam mengembangkan theCrag Topo Tool dan peta, menjadi kunci utama yang mendorong pertumbuhan dan penyebaran theCrag ke seluruh penjuru dunia.

Kini, theCrag telah menjelma menjadi platform panjat tebing dan bouldering terbesar di dunia dari segi konten. Platform ini mencatat angka fantastis: 1.409.300 rute, 8.400 lokasi panjat (crags), dan 4.498.800 pendakian yang telah dicatat. Ribuan kontributor memastikan konten terus bertambah, para sukarelawan mendukung proyek ini dengan kreativitas dan tenaga mereka, dan berbagai mitra memanfaatkannya untuk penelitian dan integrasi. Pencapaian ini jauh melampaui segala yang pernah dibayangkan di masa-masa awal theCrag.

Ingin tahu lebih banyak tentang sejarah, masa depan theCrag, dan orang-orang di baliknya? Dengarkan [\href{https://thelayback.home.blog/2020/11/28/episode-11-the-crag-building-an-enduring-resource}{Episode 11 The Layback Podcast}].

\closearticle

\section{Grades}
Sistem penilaian (grade) usianya hampir sama tuanya dengan aktivitas panjat tebing itu sendiri. Menetapkan tingkat kesulitan pada suatu jalur pendakian tampaknya sama pentingnya dengan mendeskripsikannya melalui parameter karakteristik lain, seperti panjang jalur, perlindungan (protection), atau jenis batuan.

Sistem penilaian pada awalnya dikembangkan dalam area geografis terbatas, yang sedikit memudahkan perbandingan jalur pendakian di area tersebut. Seiring waktu, sistem ini terus berkembang dan diperluas secara konstan karena para pemanjat berhasil menaklukkan jalur yang semakin sulit. Sebagai contoh, sistem penilaian yang sekarang disebut UIAA dulunya memiliki tingkat kesulitan tertinggi VI (enam). Artinya, semua jalur yang lebih sulit dari VI hanya akan diberi nilai VI, sampai akhirnya sistem ini dibuka dan diperluas pada akhir tahun 1970-an.

Area dan Gaya yang Berbeda
Bahkan hingga hari ini, banyak sistem penilaian yang terbatas pada area geografis tertentu. Beberapa terbatas pada area panjat (misalnya, Saxon, Fontainebleau), sebagian besar terbatas pada negara (seperti Afrika Selatan, Brasil, Prancis, dll.), atau bahkan benua (misalnya, Ewbanks, YDS). Namun, ada juga sistem yang diekspor dan mapan di berbagai area, membuat beberapa sistem penilaian lebih umum daripada yang lain.

Sebagian besar sistem penilaian juga spesifik pada gaya tertentu. Ada sistem penilaian untuk bouldering, untuk sport climbing, untuk aid climbing, dan sebagainya. Namun, bahkan sistem penilaian untuk gaya yang sama pun tidak selalu dapat diterjemahkan dengan baik satu sama lain. Alasannya adalah karena rentang lebar nilai (width of grades) pada skala tertentu tidak sebanding, atau karena nilai-nilai tersebut tidak linier di seluruh skala.

Tantangan Konversi dan Data Global
Bagi teman-teman sebagai pemanjat, ini berarti terkadang cukup rumit untuk secara akurat mengonversi nilai spesifik dari satu sistem penilaian ke sistem penilaian lain. Pada situs theCrag, sebagai penyedia informasi dan layanan global, mereka menampilkan semua informasi penilaian dalam konteks yang ingin dilihat pengguna, dan menggunakannya secara akurat untuk tujuan pemeringkatan (rating) dan peringkat (ranking).

Bersama dengan komunitas panjat tebing, theCrag telah dan terus mengembangkan cakupan nilai yang komprehensif dan intuitif dalam sistem mereka. Oleh karena itu, theCrag menyambut baik umpan balik tentang nilai dan perkembangan baru di area tersebut. Namun, agar hal ini bermanfaat bagi seluruh komunitas, theCrag mendorong teman-teman untuk mencatat saran dan kritik ke dalam daftar Masalah (Issues list) mereka, sehingga seluruh komunitas dapat berkontribusi secara konstruktif. [\href{https://www.thecrag.com/en/article/grades}{Lebih detail}].
\end{multicols}
\closearticle

\begin{multicols}{2}
\section{Create an Account}
Pembuatan akun pada situs \href{https://www.thecrag.com/}{theCrag} tidak begitu sulit, dapat dilakukan dengan berkunjung pada situs tersebut, kemudian pada laman \textit{Homepage} akan tersedia kotak \textit{Login} ujung samping kanan atas tab putih.\\
Dengan mengikuti instruksi pembuatan akun yang telah disediakan. Berikut sebagai contoh pengisian akun atas nama "Ed Martono".

\begin{figure}[H]
  \begin{subfigure}[c]{.48\linewidth}
    \centering
    \includegraphics[width=\linewidth]{image/Tutorial-Intro/createacc.png}%
    \caption
      {%
        Pembuatan akun%
        \label{fig:big}%
      }%
  \end{subfigure}\hfill
  \begin{tabular}[c]{@{}c@{}}
    \begin{subfigure}[c]{.48\linewidth}
      \centering
      \includegraphics[width=\linewidth]{image/Tutorial-Intro/home.png}%
      \caption
        {%
          Homepage theCrag%
          \label{fig:upper}%
        }%
    \end{subfigure}\\
    \noalign{\bigskip}%
    \begin{subfigure}[c]{.48\linewidth}
      \centering
      \includegraphics[width=\linewidth,page=2]{image/Tutorial-Intro/verif.png}%
      \caption
        {%
          Hasil verifikasi email%
          \label{fig:lower}%
        }%
    \end{subfigure}
  \end{tabular}
  \caption
    {%
      Laman depan \href{https://www.thecrag.com/}{theCrag} \& Contoh input nama, email, beserta password \& Verifikasi email setelah membuat akun%
      \label{fig:every}%
    }
\end{figure}

Dari pembuatan akun ini, teman2 dapat mengeksplor secara lebih dari fitur "\textit{\href{https://www.thecrag.com/en/article/helproot}{Help}}" terlebih dahulu untuk mendapatkan informasi lebih terkait dengan fitur dan juga pengetahuan dari theCrag sebelum menjelajah tab lain.
\begin{figure}[H]
\centering
    \begin{subfigure}[c]{.9\linewidth}
      \centering
      \includegraphics[width=\linewidth]{image/Tutorial-Intro/crags.png}%
      \caption
        {%
          List area di Indonesia%
          \label{fig:upper}%
        }%
    \end{subfigure}\\
    \begin{subfigure}[c]{.9\linewidth}
      \centering
      \includegraphics[width=\linewidth,page=2]{image/Tutorial-Intro/help.png}%
      \caption
        {%
          Kanal bantuan untuk fitur dan pengetahuan umum%
          \label{fig:lower}%
        }%
    \end{subfigure}
  \caption
    {%
      Laman depan \href{https://www.thecrag.com/}{theCrag} \& Contoh input nama, email, beserta password \& Verifikasi email setelah membuat akun%
      \label{fig:every}%
    }
\end{figure}

\end{multicols}

\begin{multicols}{2}
\section{Ticking and Logbook}
Pembuatan \textit{Log-Book} kegiatan yang sering diisi dengan tempat, bersama siapa teman2 memanjat, gaya panjat apa saja yang teman2 gunakan, berapa kali percobaan yang teman2 butuhkan untuk menyelesaikan jalur pertama pada kesulitan tertentu, atau mungkin hanya ingin mencatat jalur yang berhasil di-"\textit{send}" dalam setiap gaya, tanpa memusingkan percobaan, kegagalan, atau pemanasan. Atau juga berharap pada \textit{Logbook} yang kita buat menunjukkan performa terbaik kita kepada teman, sponsor, atau mungkin kita ingin menyimpannya secara pribadi saja, baik \textit{indoor} atau \textit{outdoor}.
Dan tentu saja, sekiranya teman2 juga tertarik pada statistik dan perkembangan teman2 dalam panjat tebing. theCrag memberikan fitur seperti Peringkat Kinerja Pemanjat (\href{https://www.thecrag.com/en/article/cpr}{Climber Performance Rating/CPR}). Berbeda dengan beberapa platform lain, di theCrag ketika teman2 melakukan "\textit{send}" dicatat pada jalur panjat yang sudah terdaftar. Hal ini memastikan bahwa semua masukan kecil yang teman2 berikan digabung untuk menghasilkan gambaran komunitas yang kuat mengenai jalur panjat tertentu dan panjat tebing di suatu area.

\subsection{Logging ascents}
theCrag menawarkan teman2 pilihan jenis catatan (tick types) terbanyak di web, memungkinkan teman2 mengubah gaya perlengkapan (\href{https://www.thecrag.com/en/article/styles}{gear style}) pemanjatan teman2, dan bahkan memungkinkan teman2 mencatat parameter pemanjatan seperti penggunaan \textit{knee pads} (pelindung lutut), proteksi yang sudah terpasang (\textit{pre-clipped}), atau perjalanan \textit{Ecopoint} teman2 ke tebing untuk membuat catatan teman2 selengkap mungkin.

\subsubsection{Tick Types} \label{TickTypes}
Panjat tebing bukan hanya tentang mencapai puncak suatu jalur, tetapi juga tentang bagaimana cara teman-teman melakukannya. Sejak tahun 1970-an, ketika Kurt Albert memperkenalkan istilah red point (lihat definisinya di bawah), banyak lagi gaya pemanjatan (yang di sini disebut sebagai jenis tick) yangka kemudian didefinisikan.

Alasan utama di balik adanya berbagai jenis tick ini adalah untuk membandingkan kualitas (dan kesulitan) dari pemanjatan yang berbeda. Sebagai contoh, sebagian besar pemanjat menganggap jelas bahwa pencapaian onsight pada suatu rute memiliki nilai lebih tinggi daripada redpoint pada rute yang sama. Atau, lebih mudah memanjat rute dengan quickdraw yang sudah terpasang (pre-clipped draws) daripada memanjat sambil memasangnya sendiri [\href{https://shorturl.at/TDfq7}{detail lebih lanjut}].

Beberapa jenis \textit{tick} berkembang secara regional, sementara yang lain terikat pada gaya perlengkapan tertentu. Seperti biasa, theCrag berusaha untuk merefleksikan realitas yang dihadapi dalam dunia panjat tebing, sehingga teman-teman diberikan banyak pilihan jenis \textit{tick} untuk dipilih, sambil tetap menyoroti yang paling umum.

theCrag, juga menyadari bahwa konsistensi jenis \textit{tick} yang berkembang secara historis belumlah sempurna. Misalnya, sementara sebagian komunitas panjat tebing menggunakan jenis \textit{tick} untuk \textit{red point} dengan perlengkapan yang sudah terpasang (\textit{pink poin}t), tidak ada pembedaan serupa untuk pemanjatan \textit{onsight} atau \textit{flash}. Selain itu, perkembangan baru dalam panjat tebing (misalnya, penggunaan pelindung lutut (\textit{knee-pads}) atau sarung tangan celah (\textit{crack gloves})) sering kali memicu diskusi tentang perbandingan pemanjatan. theCrag memungkinkan teman-teman untuk mencatat nuansa ini saat mencatat pemanjatan teman-teman, sambil tetap menggunakan jenis \textit{tick} yang sudah mapan.
Definisi untuk berbagai jenis \textit{tick} diberikan dalam tabel yang disajikan. Perlu diingat bahwa tidak setiap jenis \textit{tick} dapat diterapkan pada setiap gaya perlengkapan [\href{https://www.thecrag.com/en/article/styles}{gear style}].
\end{multicols}

\begin{center} 
\begin{longtable}{|I|L|S|M|}\hline
\textbf{Gambar}&\textbf{Nama(Istilah)}&\textbf{Style}&\textbf{Keterangan}\\\hline
\multicolumn{4}{|l|}{\textbf{Lead ascents}} \\
\multicolumn{4}{|p{16cm}|}{\small Berikut merupakan tipe \textit{ascent} yang diterapkan dalam \textit{Lead Climbing}, yang mencakup \textit{Sport Climbing}, \textit{Trad Climbing}, \textit{Ice Climbing}, dan \textit{Deep Water Soloing} (DWS)} \\
\hline
    
    % Onsight Row
    \includegraphics[scale=0.6]{image/asset/image/png/onsight.png} & Onsight & \vspace{1em}
        \Sporttag{\includegraphics[scale=.06]{image/asset/gear-style/png/gear-style-sport.png} Sport} \newline \Tradtag{\includegraphics[scale=.06]{image/asset/gear-style/png/gear-style-trad.png} Trad} \newline
        \DWStag{\includegraphics[scale=.06]{image/asset/gear-style/png/gear-style-dws.png} Deep water solo} \newline \Icetag{\includegraphics[scale=.06]{image/asset/gear-style/png/gear-style-ice.png} Ice} \vspace{1em}
    & \textit{I led this route, without falling or resting, on my first attempt without prior inspection or beta.} \\
    \hline
    
    % Flash Row
    \includegraphics[scale=0.6]{image/asset/image/png/flash.png} & Flash & \vspace{1em}
        \Sporttag{\includegraphics[scale=.06]{image/asset/gear-style/png/gear-style-sport.png} Sport} \newline \Tradtag{\includegraphics[scale=.06]{image/asset/gear-style/png/gear-style-trad.png} Trad} \newline
        \DWStag{\includegraphics[scale=.06]{image/asset/gear-style/png/gear-style-dws.png} Deep water solo} \newline \Icetag{\includegraphics[scale=.06]{image/asset/gear-style/png/gear-style-ice.png} Ice} \vspace{1em}
    & \textit{I led this route, without falling or resting, on my first attempt, but used prior inspection and/or beta.} \\
    \hline
    
    % Redpoint Row
    \includegraphics[scale=0.6]{image/asset/image/png/redpoint.png} & Red point & \vspace{1em}
        \Sporttag{\includegraphics[scale=.06]{image/asset/gear-style/png/gear-style-sport.png} Sport} \newline \Tradtag{\includegraphics[scale=.06]{image/asset/gear-style/png/gear-style-trad.png} Trad} \newline
        \DWStag{\includegraphics[scale=.06]{image/asset/gear-style/png/gear-style-dws.png} Deep water solo} \newline \Icetag{\includegraphics[scale=.06]{image/asset/gear-style/png/gear-style-ice.png} Ice} \vspace{1em}
    & \textit{I led this route, without falling or resting, but not on my first attempt (incl. repeats).} \\
    \hline
    
    % Pinkpoint Row
    \includegraphics[scale=0.6]{image/asset/image/png/pinkpoint.png} & Pink point & \vspace{1em}
        \Sporttag{\includegraphics[scale=.06]{image/asset/gear-style/png/gear-style-sport.png} Sport} \newline \Tradtag{\includegraphics[scale=.06]{image/asset/gear-style/png/gear-style-trad.png} Trad} \vspace{1em}
    & \textit{I led this route, without falling or resting, but not on my first attempt, using pre-placed gear (incl. repeats).} \\
    \hline
    
    % Greenpoint-onsight Row
    \includegraphics[scale=0.6]{image/asset/image/png/greenpointonsight.png} & Green point onsight & \vspace{1em}
        \Tradtag{\includegraphics[scale=.06]{image/asset/gear-style/png/gear-style-trad.png} Trad} \vspace{1em}
    & \textit{I onsighted a sport route using trad gear. theCrag considers this as an ascent of a trad route.} \\
    \hline
    
    % Greenpoint Row
    \includegraphics[scale=0.6]{image/asset/image/png/greenpointflash.png} & Green point flash & \vspace{1em}
        \Tradtag{\includegraphics[scale=.06]{image/asset/gear-style/png/gear-style-trad.png} Trad} \vspace{1em}
    & \textit{I flashed a sport route using trad gear. theCrag considers this as an ascent of a trad route.} \\
    \hline
    
    % Greenpoint Row
    \includegraphics[scale=0.6]{image/asset/image/png/greenpoint.png} & Green point & \vspace{1em}
        \Tradtag{\includegraphics[scale=.06]{image/asset/gear-style/png/gear-style-trad.png} Trad} \vspace{1em}
    & \textit{I led a sport route with only trad gear. theCrag considers this as an ascent of a trad route.} \\
    \hline
    
    % Hang dog Row
    \includegraphics[scale=0.6]{image/asset/image/png/dog.png} & Hang dog & \vspace{1em}
        \Sporttag{\includegraphics[scale=.06]{image/asset/gear-style/png/gear-style-sport.png} Sport} \newline \Tradtag{\includegraphics[scale=.06]{image/asset/gear-style/png/gear-style-trad.png} Trad} \vspace{1em}
    & \textit{I led this route, but rested on gear or fell on the way up. Typically used for failed attempts climbed to the top.} \\
    \hline
    
    % All-free-with-rest Row
    \includegraphics[scale=0.6]{image/asset/image/png/allfreewithrest.png} & All free with rest & \vspace{1em}
        \Sporttag{\includegraphics[scale=.06]{image/asset/gear-style/png/gear-style-sport.png} Sport} \newline \Tradtag{\includegraphics[scale=.06]{image/asset/gear-style/png/gear-style-trad.png} Trad} \vspace{1em}
    & \textit{I climbed this route all free but had a rest or a fall. This should only be used in certain regions where it is an acceptable style (e.g in Saxony). Otherwise consider using Hang dog.} \\
    \hline
    
    % Ground-up-red-point Row
    \includegraphics[scale=0.6]{image/asset/image/png/groundupredpoint.png} & Ground up red point & \vspace{1em}
        \Sporttag{\includegraphics[scale=.06]{image/asset/gear-style/png/gear-style-sport.png} Sport} \newline \Tradtag{\includegraphics[scale=.06]{image/asset/gear-style/png/gear-style-trad.png} Trad} \vspace{1em}
    & \textit{I redpointed this route, and on all prior failed attempts I immediately lowered and pulled the rope without working, resting or inspection (including on abseil) after each failed attempt.} \\
    \hline

    % Lead-solo Row
    \includegraphics[scale=0.6]{image/asset/image/png/leadsolo.png} & Lead solo & \vspace{1em}
        \Sporttag{\includegraphics[scale=.06]{image/asset/gear-style/png/gear-style-sport.png} Sport} \newline \Tradtag{\includegraphics[scale=.06]{image/asset/gear-style/png/gear-style-trad.png} Trad} \vspace{1em}
    & \textit{I led this route, using a lead solo device, anchor rope at the bottom.} \\
    \hline
\end{longtable}
\end{center}

\begin{multicols}{2}
\subsubsection{Proses pencatatan (ticking) Logbook}
\par Ambil contoh untuk kasus \textit{Gym Boulder} yang terletak di area Yogyakarta, \href{https://www.thecrag.com/en/climbing/indonesia/area/10362081504}{MAPAGAMA Boulder Wall}. Ambil contoh jalur [[ YV1 ]], kita coba Log ascent dengan klik pada nama jalur tersebut, kemudian akan muncul \textit{interface} sebagai berikut.
\begin{figure}[H]
    \centering
    \includegraphics[width=.76\linewidth]{image/Tutorial-Intro/log.png}
    \caption{Log ascent pada jalur \href{https://www.thecrag.com/en/climbing/indonesia/area/10362081504}{[[ YV1 ]]}}
    \label{fig:YV1}
\end{figure}

Dari sini kita hanya cukup untuk klik-klik pada tipe \textit{tick} yang telah kita lakukan pada boulder [[ YV1 ]] ini, dengan merujuk pada definisi \textit{Tick Types}\ref{TickTypes} sebelumnya.
\subsubsection{Climber Performance Rating (CPR)}

\par Banyak pemanjat senang mendapatkan umpan balik mengenai progres mereka, serta melihat posisi mereka dibandingkan dengan teman-teman maupun pemanjat lain di wilayah mereka atau di seluruh dunia. Meskipun kompetisi panjat tebing semakin populer dan akan dipertandingkan di Olimpiade 2020, belum ada pendekatan yang terstteman2risasi untuk sistem penilaian (rating) dan peringkat (ranking) dalam panjat tebing alam (outdoor). Beberapa situs web panjat memiliki sistem peringkat yang terkesan arbitrer (tidak berdasar) dan tidak memberikan bukti statistik atau ilmiah bagi sistem mereka, sehingga hanya memberikan manfaat yang sangat terbatas untuk mengukur dan memantau progres pribadi.\\

\par Tim theCrag berpendapat bahwa sistem penilaian bisa lebih dari sekadar cara untuk membandingkan diri teman2 dengan orang lain; theCrag juga ingin sistem ini menjadi alat bantu untuk mengukur progres kita sendiri, menginspirasi teman2 untuk memanjat pada tingkat yang lebih sulit jika itu yang menjadi tujuan teman2, serta membantu menemukan kekuatan dan kelemahan kita masing2. Sistem penilaian yang hebat harus bisa digunakan oleh pemanjat dari segala tingkat kemampuan, mulai dari atlet elit, pemula, hingga pemanjat akhir pekan biasa.\\

\par Bagi pemanjat pemula, sistem ini harus menjadi cara yang menyenangkan dan mudah untuk memantau progres, memberikan motivasi, serta membandingkan pencapaian dengan teman-teman. Bagi pemanjat elit, sistem penilaian adalah hal yang serius dan dapat berkaitan langsung dengan ketenaran serta kesuksesan finansial, sementara bagi komunitas ilmiah, ini merupakan dasar untuk penelitian terstruktur dalam bidang yang sedang terus berkembang.
\begin{figure}[H]
    \centering
    \includegraphics[width=.9\linewidth]{image/Tutorial-Intro/CPR-Lee.png}
    \caption{\href{https://www.thecrag.com/ascents/aggregate/graph-sport-cpr/has/sport-cpr/by/leecujes/}{Contoh timeline CPR dari Lee Cujes}}
    \label{fig:placeholder}
\end{figure}
\begin{itemize}
    \item \textbf{Tujuan dan Tantangan}\\
    Isi tujuan dari CPR adalah menyediakan sistem yang mencerminkan kemampuan memanjat teman2 saat ini seakurat mungkin, serta memberikan prediksi kemampuan teman2 di area yang belum diketahui. Sistem penilaian ini harus sangat didasarkan pada statistik dunia nyata dan model teoritis yang kuat mengenai kemampuan memanjat dan hubungannya dengan tingkatan kesulitan (grades). Tidak semua orang merekam pemanjatan (ticks) mereka dengan cara yang sama. Banyak pemanjat hanya mencatat keberhasilan mereka dan tidak mencatat semua percobaan (attempts) mereka. Beberapa orang hanya mencatat pemanjatan yang paling 'menarik' atau berkesan. Selain itu, setiap pemanjatan berbeda-beda; ada yang red-point, ada yang onsight, ada yang menjadi seconder, dan lain-lain. Namun, sistem harus menggunakan semua sinyal informasi yang tersedia jika memungkinkan. Jadi, tantangan pertamanya adalah menggunakan sebanyak mungkin informasi yang diberikan tanpa pernah mengurangi nilai seseorang karena data yang hilang atau tidak lengkap, atau karena cara mereka mencatat (atau tidak mencatat) percobaan dan kegagalan. Sistem ini jelas tidak bisa membaca pikiran orang, dan hanya bisa melihat riwayat catatan (tick history) mereka. Sistem juga tidak mengetahui seberapa keras mereka berlatih, apakah mereka sedang cedera, atau apakah mereka terlalu banyak makan gudeg. Pada akhirnya, sistem ini juga bergantung pada pencatatan data yang jujur. 
    
    \item \textbf{Model untuk Kemampuan Memanjat}\\
    Setelah analisis statistik yang cukup ekstensif terhadap satu juta catatan (ticks) di theCrag, berdiskusi dengan berbagai pemanjat di semua level serta para ahli di bidangnya, empat pengamatan utama telah diidentifikasi dan divalidasi sebagai fondasi CPR, sistem penilaian pemanjatan yang baru ini.
    
    \item \textbf{Faktor Kekuatan Antar Grade}\\
    Setiap grade (tingkat kesulitan) lebih sulit daripada grade sebelumnya, tetapi apa sebenarnya arti dari hal ini? Ini tidak sesederhana bahwa satu grade hanya sedikit lebih sulit dari yang terakhir. Pada dasarnya, setiap grade sebenarnya memiliki tingkat kesulitan satu urutan besaran (order of magnitude) lebih tinggi dari grade sebelumnya. Dengan kata lain, pemanjatan menjadi semakin sulit secara eksponensial saat teman2 naik grade. Ketika teman2 melihat sejumlah besar data ticks, teman2 akan melihat pola yang jelas: untuk seorang pemanjat tertentu yang baru saja mencatat pemanjatan pertama mereka di grade baru, rata-rata mereka perlu memanjat sejumlah $P$ rute dari grade di bawahnya untuk meningkatkan kemampuan mereka. Jadi, untuk setiap rute unik yang teman2 panjat, sistem memberikan teman2 kira-kira $P^g$ poin—di mana $g$ adalah grade dalam sistem penilaian internal theCrag yang sangat rinci, dengan menerapkan beberapa penyesuaian yang dijelaskan di bawah ini. Kemudian, poin-poin tersebut dijumlahkan untuk menghasilkan skor akhir teman2. Setelah itu, sistem mengambil logaritma dalam basis $P$ dan menerjemahkannya kembali menjadi grade yang bermakna. Inilah yang disebut dengan 'peringkat performa' atau Climbing Performance Rating (CPR) teman2. Melalui konsultasi dengan pemanjat elit, pelatih, dan analisis statistik, theCrag telah memilih faktor grade $P$ sebesar 5 menggunakan sistem grading Ewbank. Artinya, untuk tujuan pemeringkatan, 5 kali \textit{ascent} pada grade tertentu setara dengan satu kali \textit{ascent} pada grade berikutnya.
    \begin{figure}[H]
        \centering
        \includegraphics[width=.8\linewidth]{image/Tutorial-Intro/CPR-2.png}
        \caption{\href{https://www.thecrag.com/ascents/aggregate/graph-sport-cpr/has/sport-cpr/by/aleclandstra/}{Contoh timeline CPR dari Alec Landstra}}
        \label{fig:alec}
    \end{figure}
    \item \textbf{Memantapkan Grade Baru}\\
    theCrag membedakan dua aspek di sini: CPR dan 'grade teman2' dalam artian kemampuan memanjat. Jika teman2 baru saja mencatat rute pertama teman2 pada grade baru $X$, sebagian besar pemanjat tidak akan menganggap teman2 sebagai 'pemanjat grade $X$'. Variabilitasnya terlalu tinggi; bisa jadi itu rute yang lunak (soft), bisa jadi hanya kebetulan, dll. Namun, jika teman2 telah menyelesaikan beberapa rute di tingkat tersebut, maka ada konsensus yang jelas bahwa teman2 memang memanjat di grade itu. Oleh karena itu, sistem menerapkan penalti kecil (pengurangan nilai) sehingga teman2 memerlukan sekitar tiga catatan (ticks) pada grade tertentu agar kemampuan memanjat pribadi teman2—atau grade pribadi teman2—berada di tingkat yang sama. Namun, dalam hal CPR, setiap \textit{ascent} tetap dihitung dan setiap tick pada grade apa pun akan menambah skor. Ketentuan yang ditetapkan sistem CPR bahwa seorang pemanjat memerlukan sekitar tiga kali pendakian sukses (ticks) pada grade baru agar grade pribadinya diakui setara dengan grade tersebut (proses memantapkan grade baru).
    \item \textbf{Pergeseran Grade untuk Jenis Tick (Tick Types)}\\
    Beberapa pemanjatan adalah \textit{red-point} bersih, beberapa adalah \textit{onsight}, beberapa adalah \textit{top rope}, dan sebagainya. Bagaimana semua informasi ini dapat digunakan dan dinormalisasi sedemikian rupa sehingga dapat dibandingkan secara setara (apple-to-apple). Pada dasarnya, jenis tick yang berbeda menjadi lebih mudah atau lebih sulit karena kerja tambahan yang perlu teman2 lakukan saat memanjat. Mencari jalur (route finding) membuatnya lebih sulit, sehingga onsight lebih sulit dan bernilai lebih tinggi daripada red-point. Melakukan pink-point pada rute trad (tradisional) tidak sesulit menempatkan pengaman sendiri, jadi itu lebih mudah daripada red-point dan bahkan jauh lebih mudah daripada onsight pada rute tersebut. Jika teman2 memiliki dua rute dengan kesulitan teknis yang identik tetapi satu adalah sport dan yang lainnya diproteksi secara tradisional (trad), maka melakukan pink-point pada keduanya harusnya memiliki skor identik, tetapi melakukan onsight pada rute trad akan lebih sulit dan karenanya bernilai lebih tinggi. \href{https://www.thecrag.com/en/article/cpr}{[[Lebih lengkap]]}
\end{itemize}
\end{multicols}
\newpage
\section*{Being Contributor}
Untuk dapat menjadi editor pada situs theCrag, langkah pertama yang bisa kita lakukan adalah dengan
menghubungi Simon Dale dengan menggunakan klik tombol "\textit{Request editor permission}"
\begin{figure}[H]
  \begin{subfigure}[c]{.5\linewidth}
    \centering
    \includegraphics[width=\linewidth]{image/Tutorial-Intro/permisson.png}%
    \caption
      {%
        Menunggu Persetujuan Simon Dale%
        \label{fig:big}%
      }%
  \end{subfigure}\hfill
  \begin{tabular}[c]{@{}c@{}}
    \begin{subfigure}[c]{.48\linewidth}
      \centering
      \includegraphics[width=\linewidth,page=2]{image/Tutorial-Intro/RequestArea.png}%
      \caption
        {%
          Klik tombol \textit{Request editor permission}%
          \label{fig:lower}%
        }%
    \end{subfigure}\\
    \noalign{\bigskip}%
    \begin{subfigure}[c]{.48\linewidth}
      \centering
      \includegraphics[width=\linewidth]{image/Tutorial-Intro/reqper.png}%
      \caption
        {%
          Request untuk persetujuan editing area%
          \label{fig:upper}%
        }%
    \end{subfigure}\\
    \noalign{\bigskip}%
    \begin{subfigure}[c]{.48\linewidth}
      \centering
      \includegraphics[width=\linewidth]{image/Tutorial-Intro/Accepted.png}%
      \caption
        {%
          Apabila sudah dapat persetujuan%
          \label{fig:upper}%
        }%
    \end{subfigure}
  \end{tabular}
  \caption
    {%
      Laman depan \href{https://www.thecrag.com/}{theCrag} \& Contoh input nama, email, beserta password \& Verifikasi email setelah membuat akun%
      \label{fig:every}%
    }
\end{figure}
Apabila kita sudah mendapat izin dari Simon Dale, tombol \textit{Request editor permission} akan berubahh menjadi:
\begin{figure}[H]
  \begin{subfigure}[c]{.5\linewidth}
    \centering
    \includegraphics[width=\linewidth]{image/Tutorial-Intro/Area.png}%
    \caption
      {%
        Pilihlah are yang akan diedit%
        \label{fig:big}%
      }%
  \end{subfigure}\hfill
  \begin{tabular}[c]{@{}c@{}}
    \begin{subfigure}[c]{.48\linewidth}
      \centering
      \includegraphics[width=\linewidth,page=2]{image/Tutorial-Intro/Editing.png}%
      \caption
        {%
          Klik tombol \textit{Request editor permission}%
          \label{fig:lower}%
        }%
    \end{subfigure}
  \end{tabular}
  \caption
    {%
      Laman depan \href{https://www.thecrag.com/}{theCrag} \& Contoh input nama, email, beserta password \& Verifikasi email setelah membuat akun%
      \label{fig:every}%
    }
\end{figure}
\closearticle

\section{HELP}
Adapun untuk daftar isi yang disajikan pada situs theCrag mengacu pada laman "\textit{\href{https://www.thecrag.com/en/article/helproot}{Help}}", sebagai berikut.

\begin{multicols}{2}
\subsection{About theCrag}
\begin{itemize}
    \item \href{https://www.thecrag.com/en/article/contactus}{Contact Us}
    \item Mission and Vision
    \item Media resources
    \item Merchandise Shop
    \item Policies
    \item Supporting theCrag
    \item Supporter Benefits
    \item Perks
    \item New Features and Bugs
    \item theCrag on Topo Guru App
    \item theCrag on your Apple Watch
    \item Credit and Thanks
    \item Help for the payment process
    \item Help
\end{itemize}

\subsection{Climbing Knowledge}
\begin{itemize}
    \item Introduction to Rock Climbing
    \item Grades and Grade Conversions
    \item Route Gear Styles
    \item Tick Types
    \item Rock Types and Geology for Climbers
    \item Hardest Routes
    \item Hardest Trad Routes
    \item Hardest Boulders
    \item Climbing World Ranking
    \item Climbing Terms Glossary
\end{itemize}

\subsection{Getting Started}
\begin{itemize}
    \item Searching
    \item Grades on theCrag
    \item grAId
    \item Stars and route quality
    \item Nearby iconic routes
    \item Classic crags
    \item PDF Crag Guide
    \item Seasonality
    \item Profile Badges and Account Types
    \item Code of Etiquette
\end{itemize}

\subsection{Ticking and logbook}
\begin{itemize}
    \item Logging ascents
    \item Climber Performance Rating (CPR)
    \item CPR timeline explained
    \item Calculating tick shift
    \item Check your local and global ranking
    \item Import your logbook
    \item Export your logbook
    \item Export ascent data from 8a.nu
    \item Log Ascents For Community Managed Accounts
\end{itemize}

\subsection{Share and Connect}
\begin{itemize}
    \item Favorite crags
    \item Publish your profile
    \item Watching discussions
    \item Lists
\end{itemize}

\subsection{Adding content}
\begin{itemize}
    \item User permissions
    \item Karma
    \item Become a Data Contributor and Unlock Premium Features
    \item Warnings
    \item Adding and Editing Routes
    \item Adding and Editing Areas
    \item Naming Policy
    \item Drawing topos
    \item Taking photos for topos
    \item Hash tagging
    \item Maps and geolocations
    \item Structured tagging
    \item Private, sensitive and closed crags
    \item Creating Web-Covers
\end{itemize}
 
\subsection{Advanced Editing}
\begin{itemize}
    \item Embedding photos and videos
    \item Formatting Text
    \item Moving and sorting
    \item Merging and deleting
    \item Mixed copyright
    \item Publisher participation
\end{itemize}

\subsection{Advanced Editing}
\begin{itemize}
    \item Add your Gym to theCrag
    \item Register your Gym
    \item Indoor Gym Ascents
    \item Climbing Contests \& Competitions on theCrag
    \item Managing Gym Permissions
    \item Archiving a Gym Route
\end{itemize}

 \subsection{Partner with us}
 \begin{itemize}
     \item Advocacy Groups
     \item Crag Developers \& Guidebook Editors
     \item Local business
     \item Business sponsorship
     \item Ad specifications 
 \end{itemize}

\subsection{Embed and API}
\begin{itemize}
    \item Sharing and embeding (API)
    \item Embed theCrag
\end{itemize}
\end{multicols}